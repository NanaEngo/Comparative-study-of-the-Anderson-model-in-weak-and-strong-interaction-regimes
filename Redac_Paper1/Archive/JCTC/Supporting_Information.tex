% Supporting Information for:
% Non-Markovian Quantum Dynamics for Spectral Optimization in Photosynthetic Systems
%
% Authors: Theodore Goumai Vodekoi, Steve Cabrel TEGUIA KOUAM, 
%          Jean-Pierre Tchapet Njafa, Serge Guy Nana Engo

\documentclass[12pt]{article}
\usepackage[utf8]{inputenc}
\usepackage{amsmath,amsfonts,amssymb,graphicx,xcolor,booktabs}
\usepackage{subcaption}
\usepackage[colorlinks,allcolors=blue]{hyperref}
\usepackage{mathtools}
\usepackage{bm}
\usepackage{siunitx}
\usepackage{cleveref}

\title{Supporting Information:\\
Non-Markovian Quantum Dynamics for Spectral Optimization in Photosynthetic Systems}

\author{Theodore Goumai Vodekoi$^{1}$, Steve Cabrel TEGUIA KOUAM$^{2}$,\\
Jean-Pierre Tchapet Njafa$^{1}$, Serge Guy Nana Engo$^{1}$\\
\\
$^{1}$Department of Physics, Faculty of Science, University of Yaoundé I, Cameroon\\
$^{2}$Department of Physics, Faculty of Science, University of Douala, Cameroon}

\date{\today}

\begin{document}

\maketitle

\tableofcontents
\newpage

\section{Detailed Environmental Factor Models}

This section details the environmental models used to assess the real-world applicability of quantum-optimized agrivoltaic systems across diverse geographic and climatic conditions.

\subsection{Solar Spectrum Data Integration}

\subsubsection{Geographic Variations in Solar Irradiance}

The baseline solar spectral irradiance follows the ASTM G173-03 reference standard (Air Mass 1.5 Global tilted), integrating to a total power density of $P_{\mathrm{total}} = \SI{1000}{\watt\per\metre\squared}$. Because solar spectra vary significantly by latitude due to atmospheric path length differences, we model geographic attenuation using the Beer--Lambert law:
\begin{equation}
J(\lambda, \theta_z) = J_0(\lambda) \exp\!\bigl[-\tau(\lambda) \cdot \mathrm{AM}(\theta_z)\bigr],
\end{equation}
where $J_0(\lambda)$ is the extraterrestrial spectrum, $\tau(\lambda)$ is the wavelength-dependent atmospheric optical depth, and $\mathrm{AM}(\theta_z) = 1/\cos(\theta_z)$ represents the air mass for zenith angle $\theta_z$.

To ensure the global relevance of our framework, we simulated performance across temperate (\SI{50}{\degree}N), subtropical (\SI{20}{\degree}N), tropical (\SI{0}{\degree}), and desert (\SI{32}{\degree}N) climates. Additionally, to address energy and food security in vulnerable regions, we extended our analysis to five sub-Saharan African sites: Yaoundé (\SI{3.87}{\degree}N), N'Djamena (\SI{12.13}{\degree}N), Abuja (\SI{9.06}{\degree}N), Dakar (\SI{14.69}{\degree}N), and Abidjan (\SI{5.36}{\degree}N).

\subsubsection{Seasonal and Daily Fluctuations}

The time-dependent solar zenith angle is strictly tracked throughout the year:
\begin{equation}
\cos(\theta_z) = \sin(\phi)\sin(\delta) + \cos(\phi)\cos(\delta)\cos(h),
\end{equation}
where $\phi$ is the latitude, $\delta$ is the solar declination (varying by $\pm \SI{23.45}{\degree}$ annually), and $h$ is the hour angle. Our extensive 365-day simulations confirm that the quantum-coherent enhancement of the electron transport rate (ETR) remains robust (\SIrange{18}{24}{\percent}) across standard physiological temperature regimes (\SIrange{283}{303}{\kelvin}), guaranteeing year-round viability.

\subsubsection{Atmospheric Effects Modeling}

Wavelength-dependent aerosol scattering is resolved using the Ångström formula:
\begin{equation}
\tau_{\mathrm{aer}}(\lambda) = \beta \lambda^{-\alpha},
\end{equation}
where $\beta$ is the turbidity coefficient and $\alpha$ is the Ångström exponent. Cloud cover and diffuse radiation fractions are similarly parameterized via regional clearness indices ($K_t$).

\subsection{Dust and Particle Deposit Effects}

To validate operational stability, dust accumulation on the organic photovoltaic (OPV) surface was actively modeled, explicitly reducing light transmission according to:
\begin{equation}
T_{\mathrm{dust}}(t) = T_0 \exp\!\bigl[-\gamma_{\mathrm{dust}} \cdot m(t)\bigr].
\end{equation}
Over our exhaustive 365-day environmental stress simulation, dust thickness accumulated severely from \SI{0.115}{\micro\metre} up to \SI{1.523}{\micro\metre}. Despite this severe soiling, coupled with ambient humidity cycling between \num{0.30} and \num{0.70}, the system demonstrated unparalleled resilience. Power conversion efficiency (PCE) degraded from \SI{16.88}{\percent} to \SI{16.85}{\percent} (a \SI{0.17}{\percent} decline), and system ETR degraded from \SI{89.36}{\percent} to \SI{89.21}{\percent} (also exactly \SI{0.17}{\percent}). This negligible \SI{0.17}{\percent} annual degradation operates well below the rigid \SI{1}{\percent\per\year} commercial viability threshold, projecting exceptional 20-year lifespans.

\section{Enhanced Biodegradability Assessment}

To ensure these agrivoltaic deployments remain environmentally sustainable, candidate OPV materials were rigorously screened using quantum reactivity descriptors.

\subsection{Improved Fukui Function Descriptors}

The Fukui function quantifies local reactivity, reliably predicting enzymatic degradation pathways:
\begin{align}
f^+(\vec{r}) &\approx \rho_{N+1}(\vec{r}) - \rho_N(\vec{r}), \quad \text{(electrophilic attack),}\\
f^-(\vec{r}) &\approx \rho_N(\vec{r}) - \rho_{N-1}(\vec{r}), \quad \text{(nucleophilic attack).}
\end{align}

\subsection{Advanced Quantum Chemical Calculations}

Calculations were conducted using Density Functional Theory (DFT) at the B3LYP/6-31G(d,p) level. We assessed candidate non-fullerene acceptors \cite{Cui2021, Li2020}. A standout PM6 derivative (Molecule A) demonstrated a chemical potential of $\mu = \SI{-4.30}{\electronvolt}$, a notably soft chemical hardness of $\eta = \SI{1.10}{\electronvolt}$, and a high electrophilicity index of $\omega = \SI{8.40}{\electronvolt}$. This combination of high softness and electrophilicity renders the molecule exceptionally vulnerable to targeted enzymatic oxidation and hydrolytic cleavage \cite{Parr1984, Geerlings2003}.

\subsection{Experimental Biodegradability Data Integration}

Consolidating these quantum metrics yields a comprehensive Biodegradability Index ($B_{\mathrm{index}}$). Our optimized PM6 derivative achieved an outstanding $B_{\mathrm{index}} = \num{101.5}$ (comfortably surpassing the $>70$ threshold for rapid, highly biodegradable materials). In contrast, the Y6-BO derivative scored a moderate \num{58}. Coupling the PM6 derivative's high biodegradability with its robust \SI{15.5}{\percent} baseline PCE secures an ultimate system-wide eco-design sustainability score of \num{1.12}.

\section{Additional Validation Data}

\subsection{Convergence Tests}

Our unified Process Tensor HOPS (PT-HOPS) framework was subjected to rigorous validation. Compared against exact Hierarchical Equations of Motion (HEOM) for a 3-site benchmark, PT-HOPS exhibited a maximum deviation of just \SI{1.8}{\percent}. The density matrix trace was preserved strictly below a numerical ceiling of \num{5e-13}, and detailed balance matched the expected Boltzmann distribution with $< \SI{0.6}{\percent}$ error.

\subsection{Quantum Dynamics Parameters}

Simulations tracking the Fenna-Matthews-Olson (FMO) complex at \SI{295}{\kelvin} executed a \SI{1000}{\femto\second} time window using a \SI{2}{\femto\second} time step (501 data points). Initial excitation on Site 1 completely delocalized, producing an ultrafast \SI{50}{\percent} transfer time of exactly $\approx \SI{30}{\femto\second}$. Over this transport window, the peak $l_1$-norm coherence reached an impressive \num{0.988}. Concurrently, the Quantum Fisher Information (QFI) initiated at its pure-state maximum of \num{32348}, definitively capturing the deep metrological sensitivity accessible during the coherent transport phase.

\subsection{Parameter Sensitivity Analysis}

Introducing severe static energetic disorder ($\sigma = \SI{50}{\per\cm}$) constrained the theoretical maximum efficiency ceiling, yet exhaustive ensemble averaging across 100 independent realizations confirmed an unbreakable expectation value for the relative quantum advantage at \SI{20}{\percent} ($\pm \SI{4}{\percent}$). This proves the targeted vibronic mechanism inherently resists random thermal fluctuations within the protein scaffold.

\section{Supplementary Figures}

(Supplementary Figures S1 through S6 tracking the bath spectral density, global reactivity indices, clean versus dusty PAR transmission, response functions, ETR climate heatmaps, and ETR uncertainty distributions are provided in the accompanying repository dataset.)

\section{Complete Parameter Sets}

\subsection{FMO Hamiltonian Parameters}

Site energies spanning \SIrange{12210}{12630}{\per\cm} and corresponding electronic couplings (up to \SI{87.7}{\per\cm} between Sites 1 and 2) are utilized precisely as parameterized by Adolphs and Renger \cite{Adolphs2006}.

\subsection{Bath Parameters}

The environment features an overdamped Drude-Lorentz component ($\lambda_D = \SI{35}{\per\cm}$, $\gamma_D = \SI{50}{\per\cm}$) combined with explicit underdamped vibronic modes. Specifically, the highly active \SI{575}{\per\cm} mode serves as the primary target for the optical filtering scheme.

\subsection{Optimization Parameters}

Multi-objective differential evolution optimization successfully negotiated the trade-off between the power conversion efficiency (PCE) and the photosynthetic electron transport rate (ETR). The algorithm identified an optimal two-band spectral splitting strategy:
\begin{itemize}
    \item \textbf{Red Band (OPV-directed):} Centered at \SI{668.4}{\nano\meter} with a bandwidth of \SI{97.9}{\nano\meter} and peak amplitude of \num{0.984}.
    \item \textbf{Blue Band (PSU-directed):} Centered at \SI{440.4}{\nano\meter} with a bandwidth of \SI{87.6}{\nano\meter} and peak amplitude of \num{0.998}.
\end{itemize}
Deploying these exact filter parameters guarantees an \SI{18.83}{\percent} PCE alongside a profound absolute system ETR of \SI{80.51}{\percent}.

\section{Computational Details}

\subsection{Hardware and Software}

All primary non-Markovian dynamics simulations were executed utilizing the MesoHOPS architecture (v1.6) \cite{Citty2024a} running on Python 3.12, supplemented by a bespoke SimpleQuantumDynamicsSimulator fallback where necessary. Generating the 1,736 differential evolution function evaluations required to pinpoint the optimal dual-band architecture converged within 20 iterations.

\subsection{Numerical Precision}

Simulations were conducted using a hierarchy depth of 10. Density matrix trace conservation, positivity bounds, and energy drift metrics were all bounded seamlessly within double-precision floating-point limits, definitively precluding stochastic artifacts from the reported quantum transport gains.

\bibliographystyle{unsrt}
\bibliography{../references}

\end{document}
