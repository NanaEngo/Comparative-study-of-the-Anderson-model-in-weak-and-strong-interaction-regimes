% Conclusion Section - EES Version
% Quantum Spectral Engineering for Enhanced Agrivoltaic Efficiency

\section{Conclusion}\label{sec:Conclusion}

Spectral bath engineering enhances the photosynthetic electron transport rate by up to \SI{25}{\percent} relative to Markovian models. This enhancement, validated through HEOM benchmarking ($< \SI{2}{\percent}$ deviation), originates from non-Markovian coherence effects that extend coherence lifetimes, increase exciton delocalization, and nearly double pairwise concurrence at \SI{295}{\kelvin}. Expanded quantum metrics---including linear entropy (\SI{-38}{\percent}) and Quantum Fisher Information (\SI{+59}{\percent})---confirm that filtered states maintain quantum character throughout the energy transfer process.

Pareto frontier analysis identifies practical OPV configurations achieving \SIrange{16}{18}{\percent} power conversion efficiency with \SIrange{15}{20}{\percent} ETR enhancement through dual-band transmission at \SIlist{750;820}{\nano\meter}. Economic modelling estimates USD~\numrange{470}{3000}\,\si{ha^{-1}\,yr^{-1}} additional revenue depending on crop value, with positive returns across all climate zones studied. Geographic simulations across nine climate zones---including five sub-Saharan African sites spanning equatorial humid, tropical savanna, and Sahel climates---confirm persistent quantum advantages of \SIrange{18}{24}{\percent}, with equatorial sites benefiting from near-optimal temperature alignment and Sahel sites showing moderate aerosol-related attenuation.

These predictions are experimentally testable: ultrafast spectroscopy should detect coherence lifetime extensions under filtered illumination, while field trials should demonstrate \SIrange{10}{18}{\percent} crop productivity improvements at equivalent PV coverage. We have provided quantitative materials specifications---including evaluation of PM6 and Y6-BO derivative candidates---to guide OPV development.

Future research will prioritize: (1)~complete photosynthetic network modeling incorporating carbon fixation; (2)~experimental validation across diverse crops, particularly in sub-Saharan Africa; and (3)~adaptive filtering technologies. The spectral bath engineering principle extends beyond agrivoltaics to artificial photosynthesis and bio-inspired molecular electronics.

