% Conclusion Section - EES Version
% Spectral Bath Engineering for Quantum-Enhanced Agrivoltaics

\section{Conclusion}\label{sec:Conclusion}

We have demonstrated that spectral bath engineering—the strategic filtering of sunlight to target vibronic resonances—significantly enhances agrivoltaic performance by exploiting non-Markovian quantum dynamics. Our simulations reveal a \SI{25}{\percent} increase in the excitonic electron transport rate within the FMO complex compared to Markovian baselines. This enhancement is driven by a \SIrange{20}{50}{\percent} extension of coherence lifetimes under optimal dual-band illumination, identifying a robust quantum mechanism for increasing per-photon biological efficiency in engineered solar environments.

These molecular-scale gains translate into macroscopic agricultural advantages without sacrificing photovoltaic output. Multi-objective optimization identifies organic photovoltaic configurations that achieve an \SI{18.83}{\percent} power conversion efficiency while maintaining over \SI{80}{\percent} of the native photosynthetic rate. Simulations across diverse global climates confirm that these enhancements are geographically robust and compatible with sustainable, biodegradable materials that exhibit sub-\SI{0.2}{\percent} annual degradation.

The provided design rules and material targets offer a scalable methodology for co-optimizing solar energy generation and crop productivity. This integration of open quantum systems theory with renewable energy engineering establishes a concrete pathway for developing next-generation agrivoltaic systems that advance food and energy security simultaneously.
