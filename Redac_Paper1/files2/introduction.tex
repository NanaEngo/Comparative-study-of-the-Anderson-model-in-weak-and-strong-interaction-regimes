% Introduction Section - EES Version
% Quantum Spectral Engineering for Enhanced Agrivoltaic Efficiency

\section{Introduction}\label{sec:Introduction}

Growing demand for clean energy and food security has intensified competition for agricultural land \cite{Valle2017, Dupraz2011, Marrou2013}. Agrivoltaic systems---integrating crop production with semi-transparent photovoltaic (PV) panels---address this conflict by generating electricity and food on the same land, contributing to SDGs~2, 7 and~13 \cite{Weselek2019, Amaducci2018}. Current installations can reduce water usage by up to \num{30}\% while maintaining \num{90}\% of baseline crop yields \cite{Barron2018, Elamri2018}. However, existing designs optimise for total Photosynthetically Active Radiation (PAR) flux, treating light as a classical radiative input \cite{MaLu2025, Shugar2021}.

This approach ignores a fundamental aspect of light-harvesting: energy transfer in pigment-protein complexes is a quantum process governed by non-Markovian dynamics, where coherence and structured environmental fluctuations assist transport \cite{Engel2007, Panitchayangkoon2010, Collini2010, mohs2008, tao2020, Blankenship2011, Scholes2011, Plenio2008, Sarovar2010, Huelga2013, Rebentrost2009}. In the intermediate coupling regime typical of biological systems, Markovian approximations such as Redfield theory fail to capture critical dynamical features \cite{Ishizaki2009, Kelly2016}. Photosynthetic efficiency depends on the spectral structure of both the complex and the incident light field \cite{Curutchet2016, Gelzinis2017}.

\subsection{Quantum photosynthesis and the FMO complex}

The Fenna-Matthews-Olsen (FMO) complex of green sulfur bacteria is a well-characterised model for quantum effects in photosynthesis \cite{Fenna1975, Renger2004}. Its trimeric structure exhibits long-lived quantum coherences \cite{Engel2007, Collini2010} and serves as a standard benchmark for quantum transport \cite{Mohseni2014, Hildner2013}, with each monomer containing 7--8 bacteriochlorophyll-a molecules that funnel energy from the chlorosome antenna to the reaction centre.

Parallel advances in organic photovoltaic (OPV) technology have yielded semi-transparent devices with tuneable spectral transmission, now exceeding \num{18}\% power conversion efficiency \cite{Lunt2011, Tong2016, Zhou2019, Li2020, Cui2021}. This spectral flexibility allows for OPV materials that optimize the \textit{spectral quality} of transmitted light for photosynthesis by targeting quantum mechanical resonances.

\subsection{Spectral bath engineering}

We introduce the concept of \textit{spectral bath engineering} for agrivoltaic optimization: the deliberate modification of the photon bath experienced by photosynthetic systems through strategic spectral filtering via overlying OPV panels. In the open quantum system framework, the effective spectral density becomes $J_{\rm plant}(\omega) = T(\omega) \times J_{\rm solar}(\omega)$, where $J_{\rm solar}(\omega)$ is the solar spectral irradiance (AM1.5G standard) and $T(\omega)$ is the OPV transmission function.

We examine whether engineered $T(\omega)$ that selectively excite excitonic states quasi-resonant with vibrational modes can enhance the electron transport rate (ETR). We hypothesize that targeting specific vibronic resonances sustains electronic coherence via non-Markovian environmental memory, opening energy transfer pathways absent under broadband illumination.

This differs from classical spectral optimization, which maximizes total absorbed photon flux. Spectral bath engineering instead exploits coherence-assisted transport by shaping the spectral quality of the photon bath.

\subsection{Scope and contributions}

Using non-Markovian quantum dynamics simulations (adaptive HOPS method) with the FMO complex as a benchmark, we establish four results:
\begin{enumerate}
\item A \SI{25}{\percent} enhancement in ETR relative to Markovian models under matched photon flux, arising from vibronic resonance-assisted transport;
\item Validation through 12 independent numerical tests, including convergence against HEOM benchmarks ($< \SI{2}{\percent}$ deviation) and robustness under physiological conditions (\SI{295}{\kelvin}, $\sigma = \SI{50}{\per\cm}$);
\item Quantitative OPV design principles from Pareto frontier analysis, identifying configurations that achieve \SIrange{16}{18}{\percent} PCE with \SIrange{15}{20}{\percent} ETR enhancement;
\item Testable experimental predictions for ultrafast spectroscopy and field trials;
\item Geographic validation across nine climate zones---temperate, subtropical, tropical, desert, and five sub-Saharan African sites---confirming \SIrange{18}{26}{\percent} quantum advantages worldwide.
\end{enumerate}

Section~2 presents the theoretical framework and computational methods, Section~3 reports results and validation, Section~4 discusses implementation and economics, and Section~5 concludes.
