% Introduction Section - EES Version
% Spectral Bath Engineering for Quantum-Enhanced Agrivoltaics

\section{Introduction}\label{sec:Introduction}

Meeting growing global demand for clean energy and food security presents one of the most pressing challenges at the energy-environment nexus \cite{Valle2017, Dupraz2011, Marrou2013}. Agrivoltaic systems---integrating crop production with semi-transparent photovoltaic (PV) panels---address this challenge by generating electricity and food on the same land, contributing to UN SDGs 2 (Zero Hunger), 7 (Affordable and Clean Energy), and 13 (Climate Action) \cite{Weselek2019, Amaducci2018}. Current installations can reduce water usage by up to \SI{30}{\percent} while maintaining \SI{90}{\percent} of baseline crop yields \cite{BaronGafford2019, Elamri2018}. However, existing designs optimize for total Photosynthetically Active Radiation (PAR) flux, treating light as a classical radiative input \cite{MaLu2025, Adeh2019}.

This classical approach overlooks a fundamental aspect of light-harvesting that has direct implications for energy conversion efficiency. Energy transfer in pigment-protein complexes is a quantum process governed by non-Markovian dynamics, where coherence and structured environmental fluctuations assist transport \cite{Engel2007, Panitchayangkoon2010, Collini2010, Mohseni2008, Tao2020, Blankenship2011, Scholes2011, Plenio2008, Sarovar2010, Huelga2013, Rebentrost2009, Scholes2011, Cianci2017, Sabin2021}. In the intermediate coupling regime typical of biological systems, Markovian approximations such as Redfield theory fail to capture critical dynamical features \cite{Ishizaki2009, Kelly2016}. Photosynthetic efficiency depends on the spectral structure of both the complex and the incident light field \cite{Curutchet2016, Gelzinis2017}. This quantum mechanical perspective has direct relevance to sustainable energy technologies and could enhance the energy conversion efficiency of agrivoltaic systems.

\subsection{Quantum photosynthesis and the FMO complex}

The Fenna-Matthews-Olsen (FMO) complex of green sulfur bacteria is a well-characterised model for quantum effects in photosynthesis \cite{Fenna1975, Renger2004}. Its trimeric structure exhibits long-lived quantum coherences \cite{Engel2007, Collini2010} and serves as a standard benchmark for quantum transport \cite{Mohseni2014, Hildner2013}, with each monomer containing 7--8 bacteriochlorophyll-a molecules that funnel energy from the chlorosome antenna to the reaction centre. These quantum effects have direct implications for understanding and potentially enhancing natural photosynthesis, which is fundamentally an energy conversion process.

Parallel advances in organic photovoltaic (OPV) technology have yielded semi-transparent devices with tuneable spectral transmission, now exceeding \SI{18}{\percent} power conversion efficiency \cite{Lunt2011, Tong2016, Zhou2019, Li2020, Cui2021}. This spectral flexibility allows for OPV materials that optimize the \textit{spectral quality} of transmitted light for photosynthesis by targeting quantum mechanical resonances. This technological capability creates the opportunity to implement quantum-enhanced energy conversion systems that exploit the non-Markovian nature of photosynthetic energy transfer.

\subsection{Quantum spectral bath engineering for sustainable energy}

Building on this opportunity, we introduce the concept of \textit{quantum spectral bath engineering} for agrivoltaic optimization: the deliberate modification of the photon bath experienced by photosynthetic systems through strategic spectral filtering via overlying OPV panels. In the open quantum system framework, the effective spectral density becomes $J_{\rm plant}(\omega) = T(\omega) \times J_{\rm solar}(\omega)$, where $J_{\rm solar}(\omega)$ is the solar spectral irradiance (AM1.5G standard) and $T(\omega)$ is the OPV transmission function.

We investigate whether engineered $T(\omega)$ that selectively excites excitonic states quasi-resonant with vibrational modes can enhance the electron transport rate (ETR). We hypothesize that targeting specific vibronic resonances sustains electronic coherence via non-Markovian environmental memory, opening energy transfer pathways absent under broadband illumination. This approach represents a novel quantum engineering strategy for enhancing natural energy conversion processes, with direct implications for sustainable energy production.

This differs fundamentally from classical spectral optimization, which maximizes total absorbed photon flux. Quantum spectral bath engineering instead exploits coherence-assisted transport by shaping the spectral quality of the photon bath to maximize quantum resource utilization. This represents a new paradigm for quantum-enhanced energy conversion systems that could have applications beyond agrivoltaics.

\subsection{Environmental sustainability and eco-design integration}

While quantum advantages offer enhanced energy conversion efficiency, a critical aspect of sustainable energy technologies is environmental compatibility throughout the lifecycle. To ensure that quantum-enhanced agrivoltaic systems maintain environmental sustainability, we integrate quantum reactivity descriptors with eco-design principles. Using Fukui function analysis and global reactivity descriptors, we evaluate the biodegradability of candidate OPV materials, ensuring that quantum advantages do not come at the expense of environmental impact. This approach aligns with circular economy principles and addresses the environmental sustainability requirements of next-generation energy technologies.

\subsection{Scope and contributions}

Using non-Markovian quantum dynamics simulations (Process Tensor HOPS and Spectrally Bundled Dissipators methods) with the FMO complex as a benchmark, we establish five key results with direct implications for sustainable energy and environmental science:

\begin{enumerate}
\item A \SI{25}{\percent} enhancement in ETR relative to Markovian models under matched photon flux, arising from vibronic resonance-assisted transport that demonstrates quantum advantages in natural energy conversion;

\item Comprehensive validation through 12 independent numerical tests, including convergence against HEOM benchmarks ($< \SI{2}{\percent}$ deviation) and robustness under physiological conditions (\SI{295}{\kelvin}, $\sigma = \SI{50}{\per\cm}$), ensuring that observed effects are physically grounded;

\item Quantitative OPV design principles from Pareto frontier analysis, identifying configurations that achieve \SIrange{16}{18}{\percent} PCE with \SIrange{15}{20}{\percent} ETR enhancement, balancing electrical and biological energy conversion;

\item Eco-design integration ensuring environmental sustainability with biodegradable materials ($B_{\rm index} = 72$ for PM6 derivative), addressing lifecycle environmental impact;

\item Geographic validation across nine climate zones---temperate, subtropical, tropical, desert, and five sub-Saharan African sites---confirming \SIrange{18}{26}{\percent} quantum advantages worldwide, with particular relevance to regions where energy and food security challenges converge.
\end{enumerate}

These results establish a framework connecting quantum mechanical effects in photosynthetic energy transfer to practical agrivoltaic applications, demonstrating how quantum advantages can be engineered and validated for real-world sustainable energy systems.

Section~\ref{sec:Theory} presents the theoretical framework and computational methods, Section~\ref{sec:Results} reports results and validation, Section~\ref{sec:Discussion} discusses implementation and economics, and Section~\ref{sec:Conclusion} concludes. This work establishes a framework for quantum-enhanced sustainable energy technologies that addresses both energy conversion efficiency and environmental sustainability.
