% Results Section - EES Version
% Spectral Bath Engineering for Quantum-Enhanced Agrivoltaics

\section{Results}\label{sec:Results}

% ---------------------------------------------------------
% 1. The Physics: Main finding and mechanism
% ---------------------------------------------------------

\subsection{Quantum enhancement of electron transport rate}

Optimizing the organic photovoltaic (OPV) transmission function, $T(\omega)$, demonstrates that spectral filtering increases the photosynthetic electron transport rate (ETR) by up to \SI{25}{\percent} relative to Markovian baselines under equivalent photon flux. This gain stems from vibronic resonance-assisted transport---a non-Markovian effect inaccessible to classical intensity-based optimization.

\begin{table}[ht]
\centering
\caption{\textbf{Comparison of quantum-optimized OPV design with a state-of-the-art classical OPV design.} Both designs target comparable photovoltaic coverage; the quantum-optimized design leverages spectral bath engineering to enhance photosynthetic ETR.}
\label{tab:comparison_quantum_classical}
\begin{tabular}{lccc}
\toprule
\textbf{Design} & \textbf{PCE (\%)} & \textbf{ETR enhancement (\%)} & \textbf{Biodegradability index} \\ \midrule
Quantum-optimized OPV & 18.83 & 25.0 & 101.5 \\
Classical state-of-the-art OPV & 15.0 & 5 & 70 \\
\bottomrule
\end{tabular}
\end{table}


The maximum quantum advantage emerges when the transmitted spectrum targets the \SI{575}{\per\cm} vibronic mode via transmission windows centered at $\lambda_c \approx \SI{750}{\nano\meter}$ (\SI{13333}{\per\cm}) and $\lambda_c \approx \SI{820}{\nano\meter}$ (\SI{12195}{\per\cm}). These settings satisfy the resonance matching criterion:
\begin{equation}\label{eq:resonance_condition}
\omega_{\rm filter} \approx \omega_{\rm vibronic} \pm J_{nm}.
\end{equation}
The transmission profile selectively excites states coupled to vibrational modes, forming polaron-like states with suppressed dephasing. The non-Markovian environment subsequently sustains electronic coherence over timescales comparable to inter-site energy transfer, allowing constructive interference to accelerate transport to the reaction center.

\subsection{Coherence dynamics under spectral filtering}

To unpack the origin of this acceleration, the $l_1$-norm of coherence (\Cref{eq:l1_coherence}) demonstrates that targeted spectral filtering extends coherence lifetimes by \SIrange{20}{50}{\percent} beyond broadband illumination (\Cref{fig:coherence}). With optimal filtering, $\tau_c$ exceeds \SI{500}{\femto\second} at \SI{295}{\kelvin}, whereas broadband excitation yields $\sim$\SI{300}{\femto\second}. This prolonged coherence persists even when normalized to equivalent absorbed photon flux, verifying that the spectral contour---rather than mere intensity reduction---dictates transport efficiency. Population transfer from BChl~1 reaches \SI{50}{\percent} in approximately \SI{30}{\femto\second} (confirming ultrafast coherent funneling), producing a peak $l_1$-norm coherence of \num{0.988} inside the initial \SI{50}{\femto\second}, which decays monotonically toward zero by \SI{1000}{\femto\second} due to environment-induced decoherence. The initial Quantum Fisher Information, $F_Q = \num{32348}$ (satisfying the pure-state maximum), naturally decays to near zero as the composite state delocalizes across the FMO complex and thermalizes.

Exciton delocalization, evaluated via the inverse participation ratio $\xi_{\rm deloc}$ (\Cref{eq:IPR}), expands from $N_{\rm eff} \approx 4$ (broadband) to $N_{\rm eff} \approx 9$ (filtered). This broader spatial distribution opens supplementary quantum interference pathways to the reaction center, and crucially, survives at physiological temperatures.

Vibronic resonance matching drives this progression. Selectively addressing states quasi-resonant with discrete vibrational modes catalyzes the formation of polarons possessing tailored transfer kinetics. The ensuing dressed states undergo less dephasing because the customized filter aggressively attenuates decoherence-inducing frequencies without disrupting coherent inter-site coupling. Time-resolved population dynamics locate distinct oscillations at the vibronic mode energies, acting as a direct hallmark of sustained coherent vibronic coupling lasting hundreds of femtoseconds.

State purity $\Tr[\bm{\rho}^2]$ and von Neumann entropy $S = -\Tr[\bm{\rho} \ln \bm{\rho}]$ rigidly track the coherent-to-incoherent crossover. Broadband illumination erodes purity from $\sim \num{0.95}$ to \num{0.71} within \SI{500}{\femto\second}. Engineered filtering tempers this decay, holding purity above \num{0.82} at \SI{500}{\femto\second}---a \SI{15}{\percent} margin that mirrors the coherence lifetime extension. Likewise, von Neumann entropy drops by \SI{30}{\percent} under filtering ($S = \num{0.51}$ vs. $S = \num{0.73}$), denoting a more ordered quantum steady-state. Linear entropy, $S_L = (d/(d-1))(1 - \Tr[\bm{\rho}^2])$, parallels this restriction on state mixedness.

\begin{figure}[ht]
\centering
\includegraphics[width=0.85\textwidth]{Graphics/Figure_3.png}
\caption{\textbf{Coherence preservation and spatial delocalization mapped via spectral filtering.} (a) Time-resolved $l_1$-norm of coherence. Dual-band filtering extends the coherence lifetime by \SIrange{20}{50}{\percent} compared to the broadband baseline. (b) Inverse participation ratio ($\xi_{\rm deloc}$) confirming sustained exciton delocalization across 8--10 chromophores. (c) Protein-solvent bath spectral density, highlighting the direct overlap with targeted vibronic transitions. (d) System-bath correlation function isolating non-Markovian memory effects. Simulations conducted at physiological temperature (\SI{295}{\kelvin}) with $\sigma = \SI{50}{\per\cm}$ static disorder.}
\label{fig:coherence}
\end{figure}

\Cref{tab:quantum_metrics} quantifies the disparity between filtered and broadband illumination.

% Quantum Metrics Comparison Table
\begin{table}[ht]
\centering
\caption{\textbf{Quantitative enhancement of quantum transport metrics under selective spectral filtering.} Performance of the optimized dual-band filter (\SI{750}{\nano\meter} and \SI{820}{\nano\meter} transmission windows) benchmarked against broadband illumination. All metrics evaluated at \SI{295}{\kelvin} incorporating realistic static disorder ($\sigma = \SI{50}{\per\cm}$). Enhancements track the definitive improvement in quantum network utilization. Error margins denote \SI{95}{\percent} confidence intervals derived from 100 independent disorder realizations.}
\label{tab:quantum_metrics}
\begin{tabular}{lccc}
	\toprule
	\textbf{Metric}                        & \textbf{Filtered (\SI{750}{}/\SI{820}{\nano\meter})} & \textbf{Broadband}  &  \textbf{Enhancement}  \\ \midrule
	ETR (relative)                         &      \num{1.34 \pm 0.03}       & \num{1.00 \pm 0.02} &   \SI{+34}{\percent}   \\
	Coherence lifetime (fs)                &        \num{420 \pm 35}        &  \num{280 \pm 25}   &   \SI{+50}{\percent}   \\
	Delocalization (sites)                 &       \num{8.2 \pm 0.7}        &  \num{4.1 \pm 0.5}  &  \SI{+100}{\percent}   \\
	QFI (max)                              &       \num{12.4 \pm 1.1}       &  \num{7.8 \pm 0.8}  &   \SI{+59}{\percent}   \\
	Purity ($t = \SI{500}{\femto\second}$) &      \num{0.82 \pm 0.04}       & \num{0.71 \pm 0.05} &   \SI{+15}{\percent}   \\
	Von Neumann entropy                    &      \num{0.51 \pm 0.06}       & \num{0.73 \pm 0.07} & \SI{-30}{\percent}$^*$ \\
	Linear entropy ($S_L$)                 &      \num{0.25 \pm 0.04}       & \num{0.40 \pm 0.05} & \SI{-38}{\percent}$^*$ \\
	Pairwise concurrence                   &      \num{0.34 \pm 0.05}       & \num{0.18 \pm 0.04} &   \SI{+89}{\percent}   \\ \bottomrule
	\multicolumn{4}{l}{\scriptsize $^*$Lower entropy/linear entropy indicates more ordered quantum state (beneficial).}
\end{tabular}
\end{table}

Sustained coherence guarantees continuous delocalization, yielding the \SI{34}{\percent} empirical increase in relative ETR. An \SI{89}{\percent} jump in pairwise concurrence further indicates heavily fortified inter-site entanglement, operating in strict agreement with the vibronic resonance hypothesis.

Simultaneously, a \SI{59}{\percent} rise in Quantum Fisher Information (QFI) registers as an explicit marker of quantum-accelerated transport. Dictating parameter estimation precision via the Cram\'er-Rao bound ($\delta\theta \geq 1/\sqrt{N F_Q}$), elevated QFI confirms the system remains entrenched in a quantum coherent phase, embedding parameter-dependent information inaccessible to the broadband limit. The FMO complex's acute sensitivity to spectral bandwidth dictates that slight tuning of the overlying OPV explicitly steers the subsurface biological yield.

\begin{figure}[ht]
\centering
\includegraphics[width=0.95\textwidth]{Graphics/Quantum_Metrics_Evolution.pdf}
\caption{\textbf{Transient quantum metric evolution driving the FMO complex.} (a) Site-specific population dynamics capturing the excitation cascade from BChl~1 through the seven-chromophore network. (b) $l_1$-norm coherence trajectory, maximizing inside the first \SI{100}{\femto\second} prior to environmental suppression. (c) State purity ($\Tr[\bm{\rho}^2]$) and von Neumann entropy ($S$) marking the precise coherent-to-incoherent crossover under non-Markovian PT-HOPS dynamics at \SI{295}{\kelvin}. (d) Normalized Quantum Fisher Information ($F_Q$) quantifying the peak metrological sensitivity unlocked during the early-time coherent window.}
\label{fig:quantum_metrics_evolution}
\end{figure}

The productive temporal window for quantum-enhanced transport matches the exact regime addressed by the spectral filter (\Cref{fig:quantum_metrics_evolution}).

% ---------------------------------------------------------
% 2. Material Design: Translating the filtered requirement to OPV molecular design
% ---------------------------------------------------------

\subsection{Quantum reactivity descriptors and eco-design framework for OPV materials}

To physically instantiate these targeted filter profiles, we incorporated quantum reactivity descriptors into an eco-design loop, establishing a physics-informed basis for selecting sustainable agrivoltaic materials. The Fukui function yields a rigorous framework for assessing molecular biodegradability:
\begin{align}
f^+(\vec{r}) &= \pdv{\rho(\vec{r})}{N}_{v(\vec{r})}^+ \approx \rho_{N+1}(\vec{r}) - \rho_N(\vec{r}), \quad \text{(electrophilic attack),}\label{eq:fukui_plus_new}\\
f^-(\vec{r}) &= \pdv{\rho(\vec{r})}{N}_{v(\vec{r})}^- \approx \rho_N(\vec{r}) - \rho_{N-1}(\vec{r}), \quad \text{(nucleophilic attack),}\label{eq:fukui_minus_new}\\
f^0(\vec{r}) &= \tfrac{1}{2}\qty[f^+(\vec{r}) + f^-(\vec{r})], \quad \text{(radical attack).}\label{eq:fukui_zero_new}
\end{align}
These descriptors quantify a given molecule's susceptibility to targeted enzymatic degradation. The corresponding biodegradability index, $B_{\mathrm{index}}$, synthesizes diverse local and global reactivity metrics:
\begin{equation}\label{eq:biodegradability_index}
B_{\mathrm{index}} = w_1 S + w_2 \langle f^- \rangle + w_3 N_{\mathrm{ester}} + w_4 (400 - \mathrm{BDE}_{\mathrm{min}}),
\end{equation}
where $S$ represents global softness, $\langle f^- \rangle$ the average nucleophilic Fukui function, $N_{\mathrm{ester}}$ the number of hydrolyzable ester linkages, and $\mathrm{BDE}_{\mathrm{min}}$ the weakest bond dissociation energy in \si{\kilo\joule\per\mole}. The selected empirical weights are $w_1 = 0.3$, $w_2 = 0.3$, $w_3 = 0.2$, and $w_4 = 0.2$.

We evaluated two semi-transparent non-fullerene acceptor variants for the active layer. \textbf{Molecule~A (PM6 derivative)} demonstrates high biodegradability ($B_{\mathrm{index}} = \num{101.5}$), comfortably exceeding the standard \num{70} threshold and classifying it as highly biodegradable. This profile is driven by four hydrolyzable ester linkages and a minimum bond dissociation energy of \SI{285}{\kilo\joule\per\mole} at the thiophene--ester bond. Global reactivity descriptors map a chemical potential $\mu = \SI{-4.30}{\electronvolt}$, a low chemical hardness $\eta = \SI{1.10}{\electronvolt}$ indicating a soft molecule favorable for enzymatic oxidation, and a high electrophilicity index $\omega = \SI{8.40}{\electronvolt}$, which promotes both OPV performance and hydrolytic degradation pathways. Conversely, \textbf{Molecule~B (Y6-BO derivative)} is moderately biodegradable ($B_{\mathrm{index}} = \num{58}$), containing just two ester linkages coupled to a minimum BDE of \SI{310}{\kilo\joule\per\mole}. The PM6 derivative successfully secures a power conversion efficiency (PCE) of \SI{15.5}{\percent}. 

To critically rank these material platforms, an encompassing eco-design score scales biodegradability against life cycle assessment (LCA) impact and generation efficiency:
\begin{equation}\label{eq:eco_design_score}
\eta_{\mathrm{eco}} = 0.4 \cdot \eta_{\mathrm{biodeg}} + 0.3 \cdot \eta_{\mathrm{PCE}} + 0.3 \cdot \eta_{\mathrm{LCA}}.
\end{equation}
Operating this rubric, the optimized PM6 derivative achieves an eco-design score of $\eta_{\mathrm{eco}} = \num{1.12}$, eclipsing incumbent commercial standards in overall sustainability profile.

% ---------------------------------------------------------
% 3. System-Level Execution: Evaluating the trade-off
% ---------------------------------------------------------

\subsection{Pareto optimisation: energy versus agriculture}

Multi-objective optimization mapping the completed OPV stack generates the exact Pareto frontier governing the PCE--ETR trade-off (\Cref{fig:pareto}). Three distinct operational paradigms define the accessible design space:

The \textbf{balanced configuration} achieves an \SI{18.83}{\percent} PCE alongside an absolute system ETR of \SI{80.51}{\percent}. The differential evolution optimizer identified a two-band spectral splitting strategy comprising a primary red transmission band centered at \SI{668.4}{\nano\meter} (FWHM \SI{97.9}{\nano\meter}, amplitude \num{0.984}) tailored to the red absorption edge of the OPV, and a secondary blue transmission band at \SI{440.4}{\nano\meter} (FWHM \SI{87.6}{\nano\meter}, amplitude \num{0.998}) directed to the photosynthetic unit's Soret region. This specific splitting complements the underlying absorption profile of chlorophyll, permitting robust energy-food cogeneration.

Stepping toward grid primacy, the \textbf{energy-focused configuration} forces maximal PCE (\SI{22.1}{\percent}) at a penalty to ETR utilizing a solitary narrow band (\SI{50}{\nano\meter} FWHM). Finally, plunging into deep biostimulation, the \textbf{agriculture-focused configuration} maximizes absolute ETR while retaining minimum viable PCE (\SI{15.4}{\percent}) via two expansive transmission bands (\SI{100}{\nano\meter} FWHM).

\begin{figure}[ht]
\centering
\includegraphics[width=0.75\textwidth]{Graphics/Pareto_Front__PCE_vs_ETR_Trade_off.pdf}
\caption{\textbf{Pareto frontier resolving the energy--agriculture trade-off.} Multi-objective optimization maps the competitive boundary between standard electrical generation (PCE) and the biochemically-coupled electron transport rate (ETR). Three defining operational modes bracket the solution space: a Balanced configuration (\SI{18.83}{\percent} PCE, \SI{80.51}{\percent} system ETR), an Energy-focused peak (\SI{22.1}{\percent} PCE), and an Agriculture-focused maximum (\SI{15.4}{\percent} PCE).}
\label{fig:pareto}
\end{figure}

The frontier confirms that functional quantum advantages exist concurrently with formidable electrical capacities ($\text{PCE} \geq \SI{15}{\percent}$). Modelled across a \SI{1}{\hectare} pilot installation farming sensitive high-value produce, preserving the system ETR at \SI{80.51}{\percent} independently retains USD~\numrange{3000}{5000} in cumulative agricultural revenue annually. This dividend effectively buffers the financial transition from a \SI{22.1}{\percent} grid-tie array down to the \SI{18.83}{\percent} balanced deployment scheme.

% ---------------------------------------------------------
% 4. Environmental and Scalable Impact
% ---------------------------------------------------------

\subsection{Environmental robustness}

The coherence benefit successfully navigates broad physiological thresholds without fracturing (\Cref{fig:robustness}). The response follows a sharp non-monotonic temperature curve, showing peak coherence preservation between \SI{285}{\kelvin} and \SI{300}{\kelvin}---an interval directly matching commercial temperate agriculture. Operating under \SI{295}{\kelvin}, the quantum advantage metric $\eta_{\rm quantum}$ stabilizes at \num{0.34}; subjected to canopy heat stress (\SI{310}{\kelvin}), it safely drops to \num{0.18}. This inflection captures the physical tradeoff driving the mechanism: thermally populating active vibronic transport modes while competing against accelerated unrecoverable solvent dephasing.

Injected static energetic disorder ($\sigma = \SI{50}{\per\cm}$) curtails the maximum theoretical ceiling by roughly \SI{20}{\percent}, yet a solid macroscopic augmentation of \SIrange{18}{20}{\percent} physical endures. Exhaustive ensemble averaging yields an expectation value $\langle\eta_{\rm quantum}\rangle = \num{0.20 \pm 0.04}$, confirming a robust statistical mean. Simulating the complex under severe energetic disorder ($\sigma = \SI{100}{\per\cm}$) consistently shields a \SIrange{12}{15}{\percent} structural edge. Because intramolecular bond frequencies strictly govern the core vibronic resonance, the central pathway inherently resists random thermal fluctuations permeating the protein landscape.

Bridging simultaneous dynamic disorder (correlation times $\tau_{\rm corr} = \SIrange{50}{200}{\femto\second}$) yields final cumulative net enhancements ranging from \SI{15}{\percent} to \SI{18}{\percent}. A complete year-long environmental stress simulation (\SI{365}{\day}, tracing temperatures from \SIrange{283}{303}{\kelvin} and managing relative humidity spans of \numrange{0.30}{0.70}) resulted in remarkably low performance degradation. Over \SI{365}{\day}, with dust thickness accumulating from \SI{0.115}{\micro\meter} to \SI{1.523}{\micro\meter}, the PCE degraded by only \SI{0.17}{\percent} (from \SI{16.88}{\percent} to \SI{16.85}{\percent}) and the ETR degraded by exactly \SI{0.17}{\percent} (from \SI{89.36}{\percent} to \SI{89.21}{\percent}). Operating securely below the \SI{1}{\percent\per\yr} industry threshold assures uncompromising 20-year operational viability and outstanding environmental stability.

\begin{figure}[ht]
\centering
\includegraphics[width=\textwidth]{Graphics/ETR_Under_Environmental_Effects.pdf}
\caption{\textbf{Environmental resilience of quantum-enhanced agrivoltaics.} (a) Stability of the ETR enhancement across terrestrial temperature regimes. (b) Preservation of the quantum advantage under increasing static energetic disorder ($\sigma$). (c) Projected net performance translating site-specific insolation and ambient temperature across distinct global climatic zones. Error bars indicate \SI{95}{\percent} confidence intervals.}
\label{fig:robustness}
\end{figure}

\subsection{Geographic and climatic applicability}

Incorporating regional irradiance alongside historical thermal spectra projected comprehensive deployment performance across temperate (Germany, \SI{50}{\degree}N), subtropical (India, \SI{20}{\degree}N), tropical (Kenya, \SI{0}{\degree}), and arid (Arizona, \SI{32}{\degree}N) sectors. All mapped coordinates posted positive continuous quantum net returns structurally between \SI{18}{\percent} and \SI{26}{\percent}. Consistent baseline heat (tracking \SI{295}{\kelvin}) across tropical grids continually buoys the metrics. Arid footprints reliably process intense daytime spikes (\SIrange{305}{315}{\kelvin}) yet conserve an underlying \SIrange{15}{20}{\percent} transport boost.

Simulations extending to sub-Saharan latitudes (Yaound\'e \SI{3.87}{\degree}N; N'Djamena \SI{12.13}{\degree}N; Abuja \SI{9.06}{\degree}N; Dakar \SI{14.69}{\degree}N; Abidjan \SI{5.36}{\degree}N) demonstrate sustained enhancements spanning \SIrange{18}{24}{\percent} across humid, savanna, and Sahel biomes. Tight equatorial zones capture optimal margins mirroring strict \SIrange{297}{300}{\kelvin} profiles; marginal Sahel tracks incur minor penalties due to dense airborne particulate suspension (aerosol optical depths reaching \numrange{0.4}{0.8}) degrading precise bandpass cutoffs.

Temperate envelopes register highly regular seasonal dynamics: $\eta_{\rm quantum}$ cycles between \numrange{0.22}{0.26} during winter troughs, \numrange{0.24}{0.28} framing the equinox transitions, and rests at \numrange{0.18}{0.24} during extreme summer irradiance peaks. Evidencing uniform structural resilience irrespective of longitude or seasonal alignment, tailored passive spectral filtering establishes a reliable universal platform for integrated global food and energy co-production.

% ---------------------------------------------------------
% 5. Foundation and Validation of the Approach
% ---------------------------------------------------------

\subsection{Scalability and validation of the PT-HOPS framework}

To access these regimes concurrently simulating complex photosynthetic networks alongside explicit non-Markovian dynamics, we deployed the combined Process Tensor HOPS (PT-HOPS) and Spectrally Bundled Dissipators (SBD) framework. This platform matches the granular accuracy of traditional hierarchical equations of motion (HEOM) while decisively slashing computational overhead.

PT-HOPS functionally unwraps the bath correlation function $C(t)$ using a Padé decomposition formatted into exponentially decaying modes:
\begin{equation}\label{eq:pt_decomposition}
K_{\mathrm{PT}}(t,s) = \sum_{k} g_k(t)\, f_k(s)\, \mathrm{e}^{-\lambda_k |t-s|} + K_{\mathrm{non\text{-}exp}}(t,s),
\end{equation}
where $g_k(t)$ and $f_k(s)$ configure effective environmental couplings, $\lambda_k$ locks decay rates, and $K_{\mathrm{non\text{-}exp}}(t,s)$ safely packages residual extended memory. Activating this factorization isolates and handles explicit non-Markovian structures while ensuring unbroken scalability.

Driving networks pushing past 100 native chromophores utilizes SBD to sharply corral branching dissipative pathways:
\begin{equation}\label{eq:sbd_operator}
\mathcal{L}_{\mathrm{SBD}}[\rho] = \sum_{\alpha} p_{\alpha}(t) \mathcal{D}_{\alpha}[\rho],
\end{equation}
where $\mathcal{D}_{\alpha}[\rho] = L_{\alpha} \rho L_{\alpha}^{\dagger} - \frac{1}{2}\{L_{\alpha}^{\dagger}L_{\alpha}, \rho\}$ delineates the primary dissipator assigned to bundle $\alpha$, directly constrained by a discrete time-dependent routing probability $p_{\alpha}(t)$. Integrating this architecture scales the horizon to aggregates exceeding 1000 quantum sites without compromising embedded non-Markovian memory fidelity, successfully circumventing the $\mathcal{O}(N^3)$ processing bottleneck anchoring traditional dense HEOM stacks.

The execution toolchain cleared twelve formal validation checkpoints (Supporting Information Table~2). Results align structurally with dense full HEOM baseline controls well within \SI{1.8}{\percent} across demanding three-site configurations. Internal density matrix trace cleanly conserves below an absolute ceiling of \num{5e-13}, and processing the strict Markovian extreme limit ($T > \SI{500}{\kelvin}$) bounds deviations beneath \SI{2}{\percent}. This perfect convergence validates the internal arithmetic driving the non-Markovian engine.

Aggressive structural validation harnessed wide Monte Carlo propagation routes, sparse Latin hypercube initialization fields, and intensive bootstrap block resampling. Satisfying mathematical convergence dictated executing $10^4$ discrete evaluation passes. Verifying physical consistency demanded \num{1000} complete bootstrap resampling cycles against highly perturbed architectural parameter arrays. Recomputing a \SI{10}{\percent} fractional control set yielded coefficients of variation beneath \SI{0.5}{\percent}, precluding stochastic noise or transient artifact interference definitively.
